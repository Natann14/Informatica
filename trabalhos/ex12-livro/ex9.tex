\documentclass{article}
\usepackage[utf8]{inputenc}

\title{ex9.tex 
Funções da linguagem c}
\author{Silas Natanael Silva De Souza}
\date{August 2021}

\usepackage{natbib}
\usepackage{graphicx}

\begin{document}

\maketitle

\section{FUNÇÕES DA LINGUAGEM C}

1. FUNÇÃO MAIN - É a função principal da linguagem c que devolve um numero inteiro para informar ao sistema operacional sobre o fim da execução do programa.

2. FUNÇÃO PRINTF - A função printf () é basicamente utilizada para enviar informações ao monitor, ou seja, mostrar informações na tela.

3. FUNÇÃO SCANF - Scanf é uma família de funções da linguagem de programação C disponibilizada pelo arquivo cabeçalho stdio.h que permite a leitura de dados a partir de uma fonte de caracteres de acordo com um formato pré determinado.

4. FUNÇÃO FGETS - A função fgets (), disponível no header cstdio ou stdio.h, é usada para ler caracteres de um arquivo e armazená-los em um vetor de caracteres.

5. FUNÇÃO STRMCP - Utiliza-se a função strcmp (). A função strcmp () compara o conteúdo, ou seja, se as palavras são iguais.

6. FUNÇÃO STRLEN - Retorna o número de caracteres contidos na string str, sem contar com o caractere delimitador 0.

7. FUNÇÃO STRCAT - Recebe como parâmetro 3 strings: s1, s2, e sres. A função deve retornar em sres a concatenção de s1 e s2.

8. FUNÇÃO STRCPY - A função strcpy copia os n primeiros caracteres de uma string para outra.

9. FUNÇÃO STRTOK - Divide a string fornecida em tokens divididos pelo delimitador especificado.

10. FUNÇÃO GETCH - A função getch () lê o caractere do teclado e não permite que seja impresso na tela.

11. FUNÇÃO GETCHE - A função getche () lê o caractere do teclado e permite que seja impresso na tela.

\end{document}
